\documentclass[12pt]{book}

% Pacotes essenciais
\usepackage[utf8]{inputenc}   % Suporte a acentuação
\usepackage[T1]{fontenc}      % Codificação de fontes
\usepackage[portuguese]{babel}    % Idioma
\usepackage{lmodern}          % Fonte moderna
\usepackage{amsmath,amssymb,amsthm} % Matemática
\usepackage{graphicx}         % Imagens
\usepackage{hyperref}         % Links
\usepackage{fancyhdr}         % Cabeçalho/rodapé
\usepackage{geometry}         % Margens
\usepackage{color}            % Cores
\usepackage{quotchap}         % Estilo para citações de capítulo
\usepackage{csquotes}         % Citações
\usepackage{listings}         % Código-fonte
\usepackage{xcolor}
\usepackage{makeidx}          % Índice remissivo
\usepackage{tcolorbox}        % Quadros estilizados
\usepackage{framed}           % Caixas simples
\usepackage{enumitem}         % Listas customizáveis
\usepackage{longtable}        % Tabelas longas

\makeindex

% Configurações visuais
\geometry{a4paper, margin=3cm}
\hypersetup{
    colorlinks=true,
    linkcolor=blue,
    urlcolor=blue,
    citecolor=blue
}
\setlength{\headheight}{15pt} % Corrige aviso de fancyhdr
\pagestyle{fancy}
\fancyhf{}
\rhead{\leftmark}
\lhead{\thepage}

% Definições e Teoremas
\newtheorem{definition}{Definição}[chapter]
\newtheorem{theorem}{Teorema}[chapter]
\newtheorem{example}{Exemplo}[chapter]

% Capa personalizada
\title{\Huge\textbf{Manual de Matemática Moderna\\para Desenvolvedores}}
\author{Willams Sousa}
\date{\today}

\begin{document}

% Capa
\maketitle
\thispagestyle{empty}
\newpage

% Folha de rosto
\begin{titlepage}
  \centering
  \vspace*{4cm}
  {\Large Este é um livro de demonstração criado em LaTeX no VSCode.}\\[1cm]
  {\large Todos os direitos reservados ao autor.}\\[1cm]
  {\small \url{https://github.com/osdeving/}}
\end{titlepage}

\newpage
\tableofcontents
\newpage

\chapter{Introdução}
\section{Motivação}
Lorem ipsum dolor sit amet, consectetur adipiscing elit. Vivamus porta, purus eget laoreet scelerisque, neque libero tincidunt odio, at consequat magna justo id arcu.

Suspendisse potenti. Curabitur ac ligula nec nunc facilisis varius. Donec ut ligula id enim efficitur convallis. Sed euismod, nunc in facilisis tincidunt, nisi nisl aliquet nunc, a bibendum justo nunc non nisi.

\section{Objetivos}
\begin{itemize}
    \item Objetivo 1
    \item Objetivo 2
    \item Objetivo 3
    \item Objetivo 4
\end{itemize}

\section{Sobre o Livro}
Este livro está estruturado em capítulos independentes que exploram teoria e aplicações matemáticas.

\chapter{Máximo Divisor Comum (MDC)}

\section*{Definição}
Seja $b \neq 0$. Dizemos que $b$ é um divisor de $a$ se existe um inteiro $m$ tal que $a = mb$. Neste caso:

- $b$ é divisor de $a$;
- $a$ é múltiplo de $b$.

Utilizamos $\gcd(a, b)$ para denotar o “maior divisor comum” entre $a$ e $b$: o maior inteiro que divide ambos. Por convenção:
\begin{itemize}
  \item $\gcd(0, 0) = 0$;
  \item $\gcd(a, 0) = |a|$ para $a \neq 0$.
\end{itemize}

\section*{Definição formal}
Sejam $a$ e $b$ inteiros, e $c$ um inteiro positivo. Dizemos que $c = \gcd(a, b)$ se:
\begin{enumerate}
  \item $c$ divide $a$ e $b$;
  \item Qualquer outro inteiro que divide $a$ e $b$ também divide $c$.
\end{enumerate}

Outra forma equivalente:
\[
\gcd(a, b) = \max\{k \in \mathbb{Z}^+ \mid k \mid a \land k \mid b\}
\]

\paragraph{Propriedades:}
\begin{itemize}
  \item $\gcd(a, b) = \gcd(a, -b) = \gcd(-a, b) = \gcd(|a|, |b|)$;
  \item $\gcd(60, 24) = \gcd(60, -24) = 12$.
\end{itemize}

\paragraph{Relativamente primos:} Dois inteiros $a$ e $b$ são relativamente primos se $\gcd(a, b) = 1$.
\begin{itemize}
  \item Exemplos: $\gcd(8, 15) = 1$, pois os divisores positivos de 8 são $\{1,2,4,8\}$, e de 15 são $\{1,3,5,15\}$. O único em comum é 1.
\end{itemize}

\section*{Algoritmo de Euclides}
Para encontrar $\gcd(a, b)$, utilizamos divisões sucessivas:

\begin{enumerate}
  \item Suponha $a \geq b > 0$.
  \item Divida $a$ por $b$: \quad $a = q_1b + r_1$, \quad com $0 \leq r_1 < b$.
  \item Se $r_1 = 0$, então $\gcd(a, b) = b$.
  \item Caso contrário, divida $b$ por $r_1$: \quad $b = q_2r_1 + r_2$.
  \item Continue dividindo: $r_1 = q_3r_2 + r_3$, etc.
  \item O processo termina quando $r_n = 0$. Então:
  \[
  \gcd(a, b) = r_{n-1}
  \]
\end{enumerate}

\paragraph{Forma geral (sequência de divisões):}
\begin{align*}
  a &= q_1b + r_1 \\
  b &= q_2r_1 + r_2 \\
  r_1 &= q_3r_2 + r_3 \\
      &\vdots \\
  r_{n-2} &= q_nr_{n-1} + r_n \\
  r_{n-1} &= q_{n+1}r_n + 0
\end{align*}
\[
\therefore\quad \gcd(a, b) = r_n
\]

\paragraph{Validação:}
- O algoritmo é válido porque $r_n$ divide $r_{n-1}$, que divide $r_{n-2}$, e assim por diante até $a$ e $b$.
- Qualquer outro divisor comum também divide todos os $r_i$, então $r_n$ é o maior divisor comum.

\section*{Exemplo numérico (valores grandes)}
Queremos calcular $\gcd(1160718174,\ 316258250)$.

\begin{align*}
1160718174 &= 3 \cdot 316258250 + 211943424 \\
316258250  &= 1 \cdot 211943424 + 104314826 \\
211943424  &= 2 \cdot 104314826 + 33237872 \\
104314826  &= 3 \cdot 33237872 + 46701210 \\
\vdots & \\
...     &= ... + 1078 \\
\therefore\quad \gcd(1160718174,\ 316258250) &= 1078
\end{align*}

\paragraph{Resumo do processo:}
Cada linha da divisão segue o formato:
\[
  r_{i-2} = q_ir_{i-1} + r_i
\]
E termina quando $r_n = 0$, sendo que $r_{n-1}$ é o resultado.

\paragraph{Importância:} O algoritmo de Euclides é simples e extremamente eficiente, sendo fundamental para aplicações em criptografia, especialmente na teoria dos números e em algoritmos de chave pública como RSA.


\chapter{Fundamentos Matemáticos}
\section{Conjuntos e Funções}

\begin{definition}[Conjunto]
Um \textbf{conjunto} é uma coleção de objetos bem definidos.
\end{definition}

\begin{theorem}[União de conjuntos]
Se $A$ e $B$ são conjuntos, então $A \cup B$ também é um conjunto.
\end{theorem}

\begin{example}
Considere $A = \{1,2\}$ e $B = \{2,3\}$. Então $A \cup B = \{1,2,3\}$.
\end{example}

\section{Fórmulas matemáticas}
\begin{equation}
  \int_{0}^{\infty} e^{-x^2} dx = \frac{\sqrt{\pi}}{2}
\end{equation}

\chapter{Elementos Avançados}
\section{Listas}
\begin{enumerate}
    \item Item numerado 1
    \item Item numerado 2
\end{enumerate}

\section{Quadros}
\begin{framed}
Texto em destaque com quadro simples.
\end{framed}

\begin{tcolorbox}[colback=yellow!5!white,colframe=yellow!80!black,title=Nota]
Destaque visual com tcolorbox.
\end{tcolorbox}

\section{Código-fonte}
\lstset{
  basicstyle=\ttfamily\small,
  keywordstyle=\color{blue},
  commentstyle=\color{gray},
  stringstyle=\color{orange},
  breaklines=true,
  frame=single,
  language=Python
}

\begin{lstlisting}
def fatorial(n):
    if n == 0:
        return 1
    return n * fatorial(n-1)
\end{lstlisting}

\section{Tabelas}
\begin{table}[h]
\centering
\begin{tabular}{|l|c|r|}
\hline
Coluna A & Coluna B & Coluna C \\ \hline
Esq     & Centro    & Dir \\ \hline
\end{tabular}
\caption{Exemplo de tabela simples}
\label{tab:exemplo}
\end{table}

\section{Inserção de Imagens}
\begin{figure}[h]
    \centering
    \includegraphics[width=0.4\textwidth]{img/capa.jpg}
    \caption{Exemplo de imagem}
    \label{fig:exemplo}
\end{figure}

\section{Links externos e citações}
Veja o repositório do projeto em: \url{https://github.com/osdeving/}

\begin{displayquote}
"A matemática é o alfabeto com o qual Deus escreveu o universo." \textemdash{} Galileu Galilei
\end{displayquote}

Blah blah \cite{knuth1984}.

Este é um exemplo de nota\footnote{Aqui está a nota de rodapé.}.

Texto com palavra\index{palavra importante} marcada.

\appendix
\chapter{Códigos-fonte}
Conteúdo adicional ou implementações completas.

\chapter{Documentos adicionais}
Documentos de apoio, tabelas extensas ou materiais externos.

\printindex

\end{document}